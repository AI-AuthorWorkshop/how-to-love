\section{Related Works}

\paragraph{Psychology of Love and Relationships}
The psychology of love and relationships has been a subject of interest for many researchers. \citet{wickens2021engineering} provides an overview of engineering psychology and human performance, which can be linked to the development of loving relationships. \citet{fiske2018social} defines social psychology as the scientific study of how people's thoughts, feelings, and behaviors are influenced by the actual, imagined, or implied presence of others. This foundational work provides a basis for understanding the dynamics of love and relationships. \citet{hoang2018flow} discusses the concept of flow, which can be applied to the experience of love and the development of meaningful connections. \citet{fredrickson2001the} introduces the broaden-and-build theory of positive emotions, which suggests that the capacity to experience positive emotions, such as love, is a fundamental human strength central to human flourishing. Lastly, \citet{heider1958the} investigates the psychology of interpersonal relations, providing a foundation for understanding the mechanisms through which love and relationships develop.

\paragraph{Self-help and Empowerment}
Several studies have focused on the role of self-help and empowerment in personal development and well-being. \citet{braun2006using} presents thematic analysis as a useful and flexible method for analyzing qualitative data, which can be applied to the study of self-help and empowerment in the context of love and relationships. \citet{mitscherlich2020group} discusses the concept of group psychology, which can be relevant for understanding the dynamics of self-help groups and their role in fostering love and relationships. \citet{ms.2021microfinance} investigates the effectiveness of microfinance programs through Self-Help Groups (SHGs) in empowering women in socio-cultural and familial contexts, which can be linked to the development of loving relationships. \citet{katts2021volkskerk} explores the theology of anti-discrimination, self-help, and education in the context of the Volkskerk van Afrika, which can provide insights into the role of self-help and empowerment in fostering love and relationships. Lastly, \citet{gaikwad2023utilization} examines the utilization of earnings from SHGs for various purposes, including education and health, which can contribute to the development of loving relationships.

\paragraph{Relationship Marketing and Social Integration}
The field of relationship marketing has provided insights into the development and maintenance of long-term relationships between organizations and their customers. \citet{petersen1994the} empirically examines how ties between a firm and its creditors affect the availability and cost of funds, which can be extended to the study of interpersonal relationships in the context of love. \citet{dwyer1987developing} proposes a framework for developing buyer-seller relationships, which can be adapted to understand the dynamics of love and relationships. \citet{garbarino1999the} analyzes the relationships of satisfaction, trust, and commitment to component satisfaction attitudes and future intentions for customers, which can be applied to the study of love and relationships. \citet{froese2006cube} provides a historical review and recommendations for users about weight-length relationships, which can be relevant for understanding the dynamics of love and relationships. Lastly, \citet{birech2018the} examines the social challenges faced by widows, including social integration and decision-making, and highlights the benefits of group participation, which can be linked to the development of loving relationships.